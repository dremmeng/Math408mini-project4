% LaTeX Article Template - customizing page format
%
% LaTeX document uses 10-point fonts by default.  To use
% 11-point or 12-point fonts, use \documentclass[11pt]{article}
% or \documentclass[12pt]{article}.
\documentclass{article}

% Set left margin - The default is 1 inch, so the following 
% command sets a 1.25-inch left margin.
\setlength{\oddsidemargin}{0.25in}

% Set width of the text - What is left will be the right margin.
% In this case, right margin is 8.5in - 1.25in - 6in = 1.25in.
\setlength{\textwidth}{6in}

% Set top margin - The default is 1 inch, so the following 
% command sets a 0.75-inch top margin.
\setlength{\topmargin}{-0.25in}

% Set height of the text - What is left will be the bottom margin.
% In this case, bottom margin is 11in - 0.75in - 9.5in = 0.75in
\setlength{\textheight}{8in}
\usepackage{fancyhdr}
\usepackage{float}
\usepackage{mathtools}
\usepackage{amsmath}
\usepackage{amssymb}
\usepackage{graphicx}
\usepackage{float}
\graphicspath{ {./} }
\setlength{\parskip}{5pt} 
\pagestyle{fancyplain}
% Set the beginning of a LaTeX document
\begin{document}

\lhead{Drew Remmenga MATH 408}
\rhead{Project \#3}
%\lhead{Independent Study}
%\rhead{R Lab}

\begin{enumerate}

\item 
	\begin{equation*}
	\begin{split}
	u(t+k) & = u(t)+ku'(t)+ \frac{k^{2}}{2} u''(t) + \mathcal{O}(k^{3}) \\
	u(t+k) & = u(t)+kf(u)+ \frac{k^{2}}{2} f'(u)f(u) + \mathcal{O}(k^{3}) \\
	u(t+k) & = u(t)+\frac{k}{2}f(u)+\frac{k}{2}f(u)+ \frac{k^{2}}{2} f'(u)f(u) + \mathcal{O}(k^{3}) \\
	u(t+k) & = u(t) + \frac{k}{2}f(u)+\frac{k}{2}f(u)[ 1+ kf'(t)] + \mathcal{O}(k^{3}) \\
	u(t+k) & = u(t)  + \frac{k}{2}f(u) +  \frac{k}{2}f(u + kf(u)) + \mathcal{O}(k^{3}) \\
	U^{n+1} & = U^{n} + \frac{k}{2}f( U^{n}) +\frac{k}{2}f(U^{n} + k f(U^{n})) + \mathcal{O}(k^{3}) \\
	U^{n+1} & = U^{n} + k[\frac{1}{2}f( U^{n}) +\frac{1}{2}f(U^{n} + k f(U^{n}))] + \mathcal{O}(k^{3}) \\
	U^{n+1} & = U^{n} + k[ \frac{1}{2}V_{1} +\frac{1}{2}f(U^{n} + k V_{1})] + \mathcal{O}(k^{3}) \\
	U^{n+1} & = U^{n} + k( \frac{1}{2}V_{1} +\frac{1}{2} V_{2}) + \mathcal{O}(k^{3}) \\
	\end{split}
	\end{equation*}
\item
	\begin{enumerate}
	\item
		\begin{equation*}
		\begin{split}
		a_{ij} & = 0, i \leq j\\
		a_{1,1} &= 0, a_{1,2} = 0 \\
		a_{2,1} & \in \mathbb{R}, a_{2,2} = 0 \\
		\end{split}
		\end{equation*}
		\begin{equation*}
		\begin{split}
		\sum_{j=1}^{r} b_{j} & = 1 \\
		b_{1}+b_{2} & = \frac{1}{4}+\frac{3}{4} \\
		b_{1} +b_{2} & = 1\\
		\end{split}
		\end{equation*}
		\begin{equation*}
		\begin{split}
		\sum_{j=1}^{r} a_{ij} & = c_{i} \\
		 a_{1,1}+a_{1,2} & = c_{1} \\
		0+0 & = c_{1} \\
		a_{2,1} + a_{2,2} & = c_{2} \\
		\frac{2}{3} + 0 & = c_{2} \\
		\end{split}
		\end{equation*}
	\item
		\begin{equation*}
		\begin{split}
			U^{n+1} & = U^{n} +  \frac{k}{4} V_{1} +\frac{3k}{4} V_{2} \\
			V_{1} &= f(t,u) \\
			V_{2} & = f(t+ \frac{2k}{3}, u + \frac{2k}{3} f(t,u)) \\ 
			\tau^{n} & =\frac{k^{2}}{6}u'''(t_{n}) + \mathcal{O}(h^{3}), c_{2} = \frac{2}{3} \\ 
			U^{n+1} & = U^{n} +  \frac{k}{2} V_{1} +\frac{k}{2} V_{2} \\
			V_{1} &= f(t,u) \\
			V_{2} & = f(t+ \frac{k}{2}, u + \frac{k}{2} f(t,u)) \\ 
			\tau^{n} & = \frac{k^{2}}{6} u'''(t_{n}) + \mathcal{O}(h^{3}), c_{2} = \frac{1}{2} \\ 
		\end{split}
		\end{equation*}
		The error has the same coefficient. 
	\end{enumerate}
\item
	\begin{equation*}
	\begin{split}
	u'(t) & = f(t,u(t)), r = 1 \\
	u(t_{n+1}) - u(t_{n}) & = \int_{t_{n}}^{t_{n+1}} f(\tau,u(\tau))d\tau \\
	u(t_{n+1}) - u(t_{n}) & \approx \int_{t_{n}}^{t_{n+1}} P(\tau) d\tau \\
	& =  \int_{t_{n}}^{t_{n+1}} \sum_{i=0}^{0} L_{i}(\tau)f(t_{n+i},u(t_{n+i})) d\tau \\
	& = \int_{t_{n}}^{t_{n+1}} L_{0}(\tau)f(t_{n+1},u(t_{n+1})) d\tau \\
	& = f(t_{n+1},u(t_{n+1})) \int_{t_{n}}^{t_{n+1}} L_{0}(\tau) d\tau \\
	u(t_{n+1})  - u(t_{n}) & = f(t_{n+1},u(t_{n+1}))(t_{n+1}-t_{n}) \\
	u(t_{n+1})  & =u(t_{n})+ f(t_{n+1},u(t_{n+1}))(t_{n+1}-t_{n}) \\
	U^{n+1} & = U^{n} + k f_{n+1} \\
	\end{split}
	\end{equation*}
\item
	\begin{enumerate}
	\item
		\begin{equation*}
		\begin{split}
		r=2, \alpha_{0} = 2, \alpha_{1} = -3, \alpha_{2} = 1 \\
		\beta_{0} = \frac{-5}{12}, \beta_{1} = \frac{-20}{12}, \beta_{2} = \frac{13}{12} \\
		\sum_{j=0}^{2}\alpha_{j} = 0\\
		\sum_{j=0}^{2} j \alpha_{j} & = \sum_{j=0}^{2} \beta_{j}\\
		-1 & = -1
		\end{split}
		\end{equation*}
	\item
		\begin{equation*}
		\begin{split}
		U^{n+2} &= 3U^{n+1} - 2 U^{n} + \frac{k}{12}[-5f(t_{n}, U^{n}) -20f(t_{n+1}, U^{n+1}) + 12f(t_{n+2}, U^{n+2})], U^{n} = c_{1} + c_{2}2^{n} \\
		c_{1} + c_{2}2^{n+2} & = 3c_{1} + c_{2}2^{n+1} - 2c_{1} + -2c_{2}2^{n}  \\
		c_{1} & = c_{1} = 1  \\
		c_{2} & = 0
		\end{split}
		\end{equation*}
	\item
	We do not converge to the true values in this case. 
	\item
	Convergence $\rightarrow$ Consistency
	\end{enumerate}
\item
	\begin{enumerate}
	\item
		\begin{equation*}
		\begin{split}
		r=2, \alpha_{0} = \frac{1}{3}, \alpha_{1} = \frac{-4}{3}, \alpha_{2} = 1 , \beta_{2} = \frac{2}{3} \\
		[\frac{1}{k} \sum_{j=0}^{r} \alpha_{j}] u(t_{n}) & = 0 \\
		[\sum_{j=0}^{r} j\alpha_{j} - \beta_{j}] u'(t_{n}) & = 0 \\
		[k \sum_{j=0}^{r} \frac{j^{2}}{2} \alpha_{j} - j \beta_{j}] u''(t) & =  0 \\
		\tau(t_{n+2}) & = \mathcal{O}(h^{2}) 
		\end{split}
		\end{equation*}
\item
		\begin{equation*}
		\begin{split}
		r=3, \alpha_{1} = -1, \alpha_{3} = 1, \beta_{0} = \frac{1}{3}, \beta_{1} = \frac{-2}{3}, \beta_{2} = \frac{7}{3} \\
		[\frac{1}{k} \sum_{j=0}^{r} \alpha_{j}] u(t_{n}) & = 0 \\
		[\sum_{j=0}^{r} j\alpha_{j} - \beta_{j}] u'(t_{n}) & = 0 \\
		[k \sum_{j=0}^{r} \frac{j^{2}}{2} \alpha_{j} - j \beta_{j}] u''(t) & =  0 \\
		\tau(t_{n+2}) & = \mathcal{O}(h^{2}) 
		\end{split}
		\end{equation*}
	\end{enumerate}
\item
	\begin{enumerate}
	\item 
(i) is not consistent. \\
(ii) is not consistent.
	\item 
i
		\begin{equation*}
		\begin{split}
		p(\xi) &= 0 + -1\xi + 1\xi^{2} \\
		\xi & = 0,1
		\end{split}
		\end{equation*}
Zero stable \\
ii
		\begin{equation*}
		\begin{split}
		p(\xi) &= -2 + -1\xi + 1\xi^{2} \\
		\xi & = 2, -1
		\end{split}
		\end{equation*}
Not zero stable
	\item
	\item
	\item
	\end{enumerate}
\item
\begin{enumerate}
\item
\item
The errors trend towards $\frac{k^{2}}{6} + \mathcal{O}(h^{3})$ 
\end{enumerate}
\end{enumerate}



\end{document}
